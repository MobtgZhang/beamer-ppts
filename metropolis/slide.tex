\documentclass[12pt,aspectratio=169]{beamer}

\usetheme[progressbar=frametitle, numbering=fraction]{metropolis}
\usepackage{appendixnumberbeamer}

\usepackage{booktabs}
\usepackage[scale=2]{ccicons}

\usepackage{pgfplots}
\usepgfplotslibrary{dateplot}

\usepackage{xspace}
\newcommand{\themename}{\textbf{\textsc{metropolis}}\xspace}

% Chinese Fonts (Fontset: fandol,ubuntu)
\usepackage{ctex}

% Math Fonts
\usefonttheme{professionalfonts} 
\usepackage{mathspec}
\setsansfont[BoldFont={Fira Sans},
Numbers={OldStyle}]{Fira Sans Light}
\setmathsfont(Digits)[Numbers={Lining, Proportional}]{Fira
Sans Light}

% Change Color of the theme
\usepackage{xcolor}
\definecolor{DarkGrey}{HTML}{353535}
\definecolor{SUESRed}{RGB}{164,31,53}
\definecolor{SUESBrown}{RGB}{134,117,77}
\setbeamercolor{normal text}{ fg= DarkGrey  }
\setbeamercolor{alerted text}{ fg= SUESRed  }
\setbeamercolor{example text}{ fg= SUESBrown  }

% Bolder Fonts for presenting in a large room 
\setsansfont[BoldFont={Fira Sans SemiBold}]{Fira Sans Book}

\title{SUESBeamer模板说明}
\subtitle{教程说明}
\date{\today}
\author{MobtgZhang}
\institute{上海工程技术大学}
\titlegraphic{\hfill\includegraphics[height=2.5cm]{SUESlogo.jpg}}

\begin{document}

\maketitle

\begin{frame}{目录}
  \setbeamertemplate{section in toc}[sections numbered]
  \tableofcontents%[hideallsubsections]
\end{frame}

\section[引言\qquad Introduction]{引言}

\begin{frame}[fragile]{Metropolis主题}

  \themename 主题是Beamer主题,视觉噪音最小,灵感来自\href{https://github.com/hsrmbeamertheme/hsrmbeamertheme}{\textsc{hsrm} Beamer 主题} ,作者是Benjamin Weiss。

  通过加载以下的命令,使用Metropolis主题:
  
  \begin{verbatim}    \documentclass{beamer}
    \usetheme{metropolis}\end{verbatim}

  注意,你必须要拥有Mozilla的 \emph{Fira Sans} 字体,并且安装了XeTeX来体验这种简洁的排版。
\end{frame}
\begin{frame}[fragile]{章节}
  章节对相同主题的幻灯片进行分组处理:

  \begin{verbatim}    \section{Elements}\end{verbatim}

  \themename 为其提供了非常好的进度条指示 \ldots
  
\end{frame}

\section[标题格式 \qquad Titleformats]{标题格式}

\begin{frame}{Metropolis 标题格式}
	\themename 支持四种不同的标题格式:
	\begin{itemize}
		\item Regular
		\item \textsc{Smallcaps}
		\item \textsc{allsmallcaps}
		\item ALLCAPS
	\end{itemize}
  它们既可以对每个标题样式一次性设置也可以分别设置。
\end{frame}

\subsection{技巧}

{
    \metroset{titleformat frame=smallcaps}
    \begin{frame}{Small caps}
      这种框架使用 \texttt{smallcaps} 标题格式。

      \begin{alertblock}{潜在的问题}
        
        请注意,并不是每个字体支持small caps。如果您使用pdfTeX和Computer Modern Sans Serif字体对演示文稿进行排版,则小写的文本将改用Computer Modern Serif字体进行排版。
      
      \end{alertblock}
    \end{frame}
}

{
\metroset{titleformat frame=allsmallcaps}
\begin{frame}{All small caps}
	这个框架使用\texttt{allsmallcaps}标题格式。

	\begin{alertblock}{潜在的问题}
    由于此标题格式也使用allsmallcaps,因此您面临与\texttt{smallcaps}标题格式相同的问题。此外,这种格式可能会导致一些其他问题。如果考虑使用它,请参阅文档。

    从经验上来看,只需将其用于纯文本标题。
	\end{alertblock}
\end{frame}
}

{
\metroset{titleformat frame=allcaps}
\begin{frame}{All caps}
	此框架使用 \texttt{allcaps} 标题格式。

	\begin{alertblock}{潜在的问题}
    这种标题格式不像\texttt{allsmallcaps}格式那样有问题,但基本上也有相同的缺陷。因此,如果您想使用它,请查看文档。
	\end{alertblock}
\end{frame}
}

\section[元素\qquad Elements]{元素}

\begin{frame}[fragile]{印刷体}
      \begin{verbatim}
      
        主题为\emph{强调}文本提供了合理的默认值,
          
          \alert{强调}部分或显示\textbf{粗体}结果。
      
        \end{verbatim}

  \begin{center}变成以下的格式\end{center}

  主题为\emph{强调}文本提供了合理的默认值,\alert{强调}部分或显示\textbf{粗体}结果。
  
\end{frame}

\begin{frame}{字体样式测试}
  \begin{itemize}
    \item 正常体
    \item \textit{斜体}
    \item \textsc{SmallCaps}
    \item \textbf{加粗}
    \item \textbf{\textit{斜体加粗}}
    \item \textbf{\textsc{粗体 SmallCaps}}
    \item \texttt{Monospace}
    \item \texttt{\textit{Monospace 斜体}}
    \item \texttt{\textbf{Monospace 加粗}}
    \item \texttt{\textbf{\textit{Monospace 加粗斜体}}}
  \end{itemize}
\end{frame}

\begin{frame}{列表}
  \begin{columns}[T,onlytextwidth]
    \column{0.33\textwidth}
      列表 Items
      \begin{itemize}
        \item Milk \item Eggs \item Potatos
      \end{itemize}

    \column{0.33\textwidth}
      枚举 Enumerations
      \begin{enumerate}
        \item First, \item Second and \item Last.
      \end{enumerate}

    \column{0.33\textwidth}
      描述 Descriptions
      \begin{description}
        \item[PowerPoint] Meeh. \item[Beamer] Yeeeha.
      \end{description}
  \end{columns}
\end{frame}
\begin{frame}{动画}
  \begin{itemize}[<+- | alert@+>]
    \item \alert<4>{这个\only<4>{ 真的} 重要}
    \item 现在是这个
    \item 然后现在是这个
  \end{itemize}
\end{frame}
\begin{frame}{Figures}
  \begin{figure}
    \newcounter{density}
    \setcounter{density}{20}
    \begin{tikzpicture}
      \def\couleur{alerted text.fg}
      \path[coordinate] (0,0)  coordinate(A)
                  ++( 90:5cm) coordinate(B)
                  ++(0:5cm) coordinate(C)
                  ++(-90:5cm) coordinate(D);
      \draw[fill=\couleur!\thedensity] (A) -- (B) -- (C) --(D) -- cycle;
      \foreach \x in {1,...,40}{%
          \pgfmathsetcounter{density}{\thedensity+20}
          \setcounter{density}{\thedensity}
          \path[coordinate] coordinate(X) at (A){};
          \path[coordinate] (A) -- (B) coordinate[pos=.10](A)
                              -- (C) coordinate[pos=.10](B)
                              -- (D) coordinate[pos=.10](C)
                              -- (X) coordinate[pos=.10](D);
          \draw[fill=\couleur!\thedensity] (A)--(B)--(C)-- (D) -- cycle;
      }
    \end{tikzpicture}
    \caption{来自\href{http://www.texample.net/tikz/examples/rotated-polygons/}{texample.net}网站提供的旋转正方形。}
  \end{figure}
\end{frame}
\begin{frame}{表格}
  \begin{table}
    \caption{世界上最大的城市(来源:Wikipedia)}
    \begin{tabular}{lr}
      \toprule
      城市 & 人口\\
      \midrule
      墨西哥城 & 20,116,842\\
      上海 & 19,210,000\\
      北京 & 15,796,450\\
      伊斯坦布尔 & 14,160,467\\
      \bottomrule
    \end{tabular}
  \end{table}
\end{frame}
\begin{frame}{模块}
  预定义了三种不同的块环境,可以使用可选背景色。

  \begin{columns}[T,onlytextwidth]
    \column{0.5\textwidth}
      \begin{block}{默认}
        这是默认的模块
      \end{block}

      \begin{alertblock}{提醒}
        这是提醒模块
      \end{alertblock}

      \begin{exampleblock}{例子}
        这是例子模块
      \end{exampleblock}

    \column{0.5\textwidth}

      \metroset{block=fill}

      \begin{block}{默认}
        默认的文本块
      \end{block}

      \begin{alertblock}{提醒}
        提醒的文本块
      \end{alertblock}

      \begin{exampleblock}{例子}
        例子文本块
      \end{exampleblock}

  \end{columns}
\end{frame}
\begin{frame}{数学公式}
    \metroset{block=fill}
    \begin{theorem}
    微积分基本公式:$\int_a^b f(x)\mathrm{d}x=F(b)-F(a)$。
    \end{theorem}
    \begin{proof}
    令 $g(x)=e^x-x-1$。则当 $x>1$ 时, 有 $g'(x)=e^x-1>0$,
    因此 $g(x)>g(1)=0$。即有 $x>1$ 时 $e^x>1+x$。
    \end{proof}
  \begin{equation*}
    e = \lim_{n\to \infty} \left(1 + \frac{1}{n}\right)^n
  \end{equation*}
\end{frame}
\begin{frame}{折线图}
  \begin{figure}
    \begin{tikzpicture}
      \begin{axis}[
        mlineplot,
        width=0.9\textwidth,
        height=6cm,
      ]

        \addplot {sin(deg(x))};
        \addplot+[samples=100] {sin(deg(2*x))};

      \end{axis}
    \end{tikzpicture}
    \caption{这是一个折线图}
  \end{figure}
\end{frame}
\begin{frame}{条形图}
  \begin{figure}
    \begin{tikzpicture}
      \begin{axis}[
        mbarplot,
        xlabel={不同区间分布},
        ylabel={数据量个数},
        width=0.9\textwidth,
        height=6cm,
      ]

      \addplot plot coordinates {(1, 20) (2, 25) (3, 22.4) (4, 12.4)};
      \addplot plot coordinates {(1, 18) (2, 24) (3, 23.5) (4, 13.2)};
      \addplot plot coordinates {(1, 10) (2, 19) (3, 25) (4, 15.2)};

      \legend{lorem, ipsum, dolor}

      \end{axis}
    \end{tikzpicture}
    \caption{这是一个示例的统计图}
  \end{figure}
\end{frame}
\begin{frame}{引用}
  \begin{quote}
    这里是一个引用标签,可以使用引用的方式来表示一些内容。
  \end{quote}
\end{frame}

{%
\setbeamertemplate{框架页脚}{我的自定义页脚}
\begin{frame}[fragile]{框架页脚}
    \themename defines a custom beamer template to add a text to the footer. It can be set via
    \themename 定义了自定义beamer模板以向页脚添加文本。它可以通过
    \begin{verbatim}\setbeamertemplate{框架页脚}{我的自定义页脚}\end{verbatim}

    来定义页脚。
\end{frame}
}

\begin{frame}{参考文献}
  对showcase的一些引用 [allowframebreaks] \cite{knuth92,ConcreteMath,Simpson,Er01,greenwade93}
\end{frame}

\section{结论}

\begin{frame}{总结}

  可以通过以下的方式获取此主题和演示文稿的来源。

  \begin{center}\url{https://github.com/MobtgZhang/beamer-ppts}\end{center}

  此模板来源于\href{https://github.com/matze/mtheme}{mtheme}

  \begin{center}\ccbysa\end{center}

\end{frame}

{\setbeamercolor{palette primary}{fg=black, bg=yellow}
\begin{frame}[standout]
  问题提问?
\end{frame}
}

\appendix

\begin{frame}[fragile]{备用页面}
  有时,参考听众提问过程中,可以在演示结束部分添加幻灯片。

  这个最好的做法是在备用页面之前将\verb|appendixnumberbeamer|
  包含在你的序言中并且调用\verb|\appendix|。

  \themename 将自动关闭幻灯片编号和进度条附录中的幻灯片。
\end{frame}

\begin{frame}[allowframebreaks]{参考文献}

  \bibliography{slide}
  \bibliographystyle{abbrv}

\end{frame}

\end{document}
